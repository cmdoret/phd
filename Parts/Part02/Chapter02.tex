% Chapter X

\chapter{Infection of \textit{Acanthamoeba castellanii} by \textit{Legionella pneumophila}} % Chapter title

\label{ch:02-02} % For referencing the chapter elsewhere, use \autoref{ch:name} 

%----------------------------------------------------------------------------------------

Legionella pneumophila alters its host signal transduction, metabolism and gene regulation upon infection. In addition to all these changes, it also affects host histone marks, which are known to be related to gene regulation and genome architecture. In this chapter, we investigate the genome structure of \textit{A. castellanii} and how it is affected during infection by \textit{L. pneumophila}.

\section{Genome assembly}

As with most other genomics techniques, a prerequisite of Hi-C analyses is to have a high quality reference genome with clearly delimited chromosomes. At the time of writing, the A. castellanii reference genome is split into 384 scaffolds which do not represent chromosomes. This prompted us to generate a chromosome-level genome assembly.

\section{Genome architecture of \textit{A. castellanii}}

\section{Strains comparison}

Several strains of \textit{A. castellanii} have been isolated throughout history.These strains may originate from different ecological niche of geographical location and have been cultivated in labs for long preriods. As a result, they can differ in various phenotypes, including susceptibility to infection. Comparing such divergent strains can help us understand what genomic features are important for pathogen susceptibility.


\section{Changes during infection}
