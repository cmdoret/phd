% Chapter X

\chapter{Infection of \textit{Acanthamoeba castellanii} by \textit{Legionella pneumophila}} % Chapter title

\label{ch:02-02} % For referencing the chapter elsewhere, use \autoref{ch:name} 

%----------------------------------------------------------------------------------------

Legionella pneumophila alters its host signal transduction, metabolism and gene regulation upon infection. In addition to all these changes, it also affects host histone marks, which are known to be related to gene regulation and genome architecture. In this chapter, we investigate the genome structure of \textit{A. castellanii} and how it is affected during infection by \textit{L. pneumophila}.

Several strains of \textit{A. castellanii} have been isolated throughout history.These strains may originate from different ecological niche of geographical location and have been cultivated in labs for long preriods. As a result, they can differ in various phenotypes, including susceptibility to infection. Comparing such divergent strains can also help us understand what genomic features are important for pathogen susceptibility.

As with most other genomics techniques, a prerequisite of Hi-C analyses is to have a high quality reference genome with clearly delimited chromosomes. At the time of writing, the A. castellanii reference genome is split into 3192 contigs merged into 384 scaffolds which do not represent chromosomes.

This prompted us to generate a chromosome-level genome assembly for two strains of \textit{A. castellanii}: The Neff strain \cite{neffPurificationAxenicCultivation1957}, which is the reference strain for genomic analyses in that species, and the C3 strain \cite{michelIsolationAcanthamoebaStrain1997}, which is generally used for infection experiments with \textit{L. pneumophila} due to higher infection rates. In the following manuscript, we describe and compare the genomic landscape of both strains, as well as the spatial organization of their genomes. We also investigate changes happening in the 3D genome organization in response to infection by \textit{L. pneumophila}.

\includepdf[pages=-,addtotoc={
     1,section,1,Regulation of the \textit{A. castellanii} genome upon infection,p1,
     1,subsection,2,Introduction,p2,
     2,subsection,2,Results,p3,
     7,subsection,2,Discussion,p4,
     7,subsection,2,Methods,p5}]
     {Publications/acastellanii_legionella_manuscript.pdf}    

