% Chapter X

\chapter{Extracting information from Chromosome contact maps} % Chapter title

\label{ch:02-01} % For referencing the chapter elsewhere, use \autoref{ch:name} 

%----------------------------------------------------------------------------------------

Most genomics methods generate a large amount of information, most of which is not directly relevant for the problem at hand. One of the main challenges emanating from genomics data is to distill this information and extract only the relevant signal.

In the case of Hi-C and other 3D genomics techniques, the resulting signal is a list of contacts between pairs of genomic regions. These contacts reflect the average genome structure from a population of cells and are subject to various biases.

The spatial features and changes of interest are diluted in the population and can be obfuscated by noise. Detecting these changes requires a set of bias correction and signal detection methods which are still in their early developments.

In this section we review the recent methodlogical developments that allow to correct the Hi-C signal and present new methods to extract biological features from these datasets. These developments proved necessary to tackle the questions raised in further chapters.

\section{Streamlining pipelines}
% Existing methods: painful to install, non tested (risky), non reproducible
% hicstuff, hicreppy

The pre-processing of Hi-C data itself, to convert \acrfull{NGS} reads into chromosomal contact matrices involves several steps that will impact the resulting signal.

The sequencing reads themselves can be the result from religation of two distinct loci. These chimeric reads cannot be aligned reliably with generic methods and need to be cut for proper alignment. Chimeric reads become more problematic when increasing the read size relative to restriction fragment length.

Not all read pairs generated by Hi-C experiments represent valid spatial interactions. Some restriction fragments are sequenced without religation and other fragments religate on themselves. The various interaction types can be separated based on the strand of origin of their individual reads. In theory, and in practice at long ranges, one would expect religations to be strand agnostic and to have an equal represenation of all four possible combinations (++, --, +-, -+). In reality, this is never the case at short range contacts, due to the enrichment of self-religation (-+) and dangling ends (or undigested fragments, +-).

These biases must be accounted when processing Hi-C data. This can be achieved by identifying and filtering out faulty interactions based on their strands.

This preprocessing is often performed using custom scripts and prone to errors, butgs and lack of informations about parameters. In an effort to improve reproducibility and accessibility of Hi-C analysis, we developed hicstuff, an open source Hi-C pipeline that incorporate all the aforementioned steps, along with several downstream processing utilities.

Hicstuff is meant to be easily accessible, even to non-expert users. It has a comprehensive online documentation and tutorials and the program and its dependencies are installed with a single command. The code is written in python, and is covered by unit tests to reduce the likelihood of bugs. Hicstuff runs well with default parameters, but has many options to fit most common use cases. It works regardless of genome size organism.

The pipeline also provides reproducibility through an automatic logging of every intermediate result in the pipeline as well as the input parameters used.

The project has already fostered a community of users which are offering their contributions, suggest features or report issues they encounter.

\section{Pattern detection in chromosome contact maps}



\section{Change detection across biological conditions}

Change detection is a common issue in the field of signal processing and remote sensing. Given two or more input signals, such as images, we want to find portions that differ between the two inputs. This principle can also be applied to Hi-C contacts, where we can detect regions of contact maps that differ between biological conditions.

Many approaches can be taken to detect these changes. Some of them, such as diffhic, formulate the problem similarly to a differential expression RNAseq analysis using contact counts instead of read counts. This approach has the benefit of being statistically sound and straightforward, but it only finds contact increase, which could be caused by a number of different phenomenon.

We developed pareidolia, a method with an apriori on the type of signal to detect. The method is "supervised" in the sense that it requires a kernel representing the feature of interest. Pareidolia relies on Chromosight's backend to convert the contact map of each condition into a map of correlation coefficients representing similarity with the feature of interest. Change detection is then performed on these coefficients. As a consequence, rather than looking for contacts increase, pareidolia looks for changes in feature intensity, such as border sharpness or looping intensity.
