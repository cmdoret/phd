\ctpartquote{}
\ctparttext{}
\part{Appendices}

\chapter{Publications}
 \label{ch:04-A:publications}
    Articles published in the context of this work are included below.

    \includepdf[
        pages=-,
        addtotoc={
            1,
            section,
            3,
            Computer vision for pattern detection in chromosome contact maps,
            chromosight_paper
        }
    ]{Publications/chromosight_publication_supp.pdf}

\chapter{Supplementary information}
\label{ch:04-B:supdata}

    \section{Sparse convolution in Chromosight}
    \label{sec:04-B-01:convolution}

The explanation below describes how Chromosight reformulates convolution into a matrix multiplication problem to better handle large sparse matrices. The algorithm is inspired from \cite{NeuralNetwork2D}. Let S be the signal (Hi-C) matrix and K the kernel matrix.

\begin{equation}
    S = 
    \begin{bmatrix}
        4 & 2 & 1 \\
        2 & 4 & 1 \\
        1 & 1 & 3
    \end{bmatrix}
    K =
    \begin{bmatrix}
        10 & 12 \\
        11 & 13 \\
    \end{bmatrix}
\end{equation}

The dimensions of the desired convolution output are defined by:

\begin{equation}
    (m_S - m_K + 1) \times (n_S - n_K + 1)
\end{equation}

Note this corresponds to a convolution in "valid" mode, where edge values are truncated.

We transform each column of the kernel into a Toeplitz matrix with the same number of columns as the input signal. This matrix is square, and each value along the diagonals is constant.

\begin{align}
    T_0 &=
    \begin{bmatrix}
        10 & 11 & 0 \\
        0  & 10 & 11
    \end{bmatrix} &
    T_1 &=
    \begin{bmatrix}
        12 & 13 & 0 \\
        0  & 12 & 13
    \end{bmatrix}
\end{align}

The convolution of the signal and kernel can now be replaced by a sum of dot products between the signal and filter Toeplitz matrices.

\begin{align}
    C &= S * K \\
      &= S \cdot T_0 + S \cdot T_1
\end{align}

Where $\cdot$ is the matrix dot product operator and $*$ is the convolution operator.