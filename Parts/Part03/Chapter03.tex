% Chapter X

\chapter{Perspectives} % Chapter title

\label{ch:03-03} % For referencing the chapter elsewhere, use \autoref{ch:name} 

As 3C protocols improve, as I have observed during the course of the last three years, and the cost of sequencing decreases, it becomes possible to probe finer details of spatial regulation during bacterial infection. While current projects are mostly limited to analyzing major changes, higher sequencing depth and increasing numbers of replicates will allow for more contrast and with it, the detection of more subtle changes in spatial interactions \cite{mullerCharacterizingMeioticChromosomes2018}.

% Single cell (spatial heterogeneity)
Another exciting perspective is the advent of single-cell omics methods. This is especially interesting for infection genomics, where bulk Hi-C signal contains a mixture of cells at different infection stage and cell cycle phase. These single-cell methodologies may allow to further refine the analysis and deconvolute different effects obscuring the signal of interest.

In future years, we expect to see major developments in the use of 3D genomics to understand the deregulation induced by infection. There is still much to be learnt in the interplay of the various layers of regulation, and spatial organization will likely become an integrative part of many projects aiming to understand it. This work allowed us to observe the general chromosomal biology of \textit{A. castellanii}, but it would be interesting to study the behaviour of specific features in more details, such as the role of subtelomeric or rDNA clustering. This will also require additional effort to resolve repeated sequences in the assembly. Exploring different infection systems, such as different amoebae hosts, bacteria, or even megaviruses would also provide more insights into whether there are conserved hallmarks of spatial chromosomal changes during infection.

More generally, as more infection studies integrate Hi-C data with epigenetic marks and expression, it will be interesting to study in more details the interplay of structural changes with regulatory information to better understand the factors determining the importance of long range interactions in gene regulation.

%----------------------------------------------------------------------------------------
