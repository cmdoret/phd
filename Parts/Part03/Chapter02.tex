% Chapter X

\chapter{Perspectives of genomics for infection biology} % Chapter title

\label{ch:03-02} % For referencing the chapter elsewhere, use \autoref{ch:name} 

%----------------------------------------------------------------------------------------

\section{The 3D genome and the advent of deep learning}
% Deepnog (annotation)
% DeepHiC, SRHic, ...(resolution)
% akita & deepc
In some cases, it may be attractive to produce a model of the 3D genome, be it to predict its structure, or how it reacts to specific changes. However, in many organisms, the rules governing genome organization are intricate and it would be unwieldly to model them explicitely. Deep learning provides an attractive framework to produce such a model without knowing all the rules involved. There are already successful applications of deep learning in biology for various different tasks such as gene annotation \cite{stiehlerHelixerCrossspeciesGene2020}, variant calling \cite{poplinUniversalSNPSmallindel2018}, classification of coding RNA \cite{hillDeepRecurrentNeural2018} and perhaps most importantly, protein structure prediction \cite{jumperHighlyAccurateProtein2021}.

%cite akita and deep3C
Generally, applications of deep learning in the 3D genome field have been limited to denoising or improving the resolution of Hi-C matrices. Recently however, there have been successful attempts at predicting the structure of mammalian genomes from the DNA sequence, and even predict conformational changes induced by mutations. In the future, these approaches could be helpful to select mutants, or regions to focus on, and one could imagine it being used to model the consequences of infection on the host.

Unfortunately, several limitations that must be overcome before deep learning methods can become useful to understand the relationship between biological processes and genome structure. First, it requires enormous amounts of training data, which in the case of Hi-C is very expensive to generate. Then, in the event that such a model could effectively predict conformational changes, in most cases this is insufficient. The end goal is usually to understand the rules and factors that connect the biological process (e.g. infection) to structural changes. In deep learning models these rules are obscured, taking the form of large weight matrices, and extracting them would require consequent advances in model interpretability \cite{talukderInterpretationDeepLearning2021}.
