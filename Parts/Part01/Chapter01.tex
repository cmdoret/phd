% Chapter X

\chapter{Host parasite interactions} % Chapter title

\label{ch:01-01} % For referencing the chapter elsewhere, use \autoref{ch:name} 


A large number of organisms throughout the tree of life establish stable interactions with different species. These interactions are often classified according to their perceived impact on the fitness of their members. We traditionally talk about parasitism for interactions with one-way benefits, and mutualism when the interaction has a positive impact on all parties involved. Rather than a dichotomous classification, the difference between parasitism is better viewed as a spectrum, depending on the relative benefit and cost of a relationship.

Biological interactions are observed at different scales, from nanometer-scale virophages infecting giant viruses to fungi forming mycorrhyzal networks spanning several meters \citep{Johnson1997,Selosse2006} allowing exchange of nutrients with plants root systems. These interactions shape the evolutionary trajectories of the species involved and their genomic landscapes. These changes can sometimes result in drastic transitions in the organisms' lifestyle. 

This can be the case for example with intracellular bacteria forming symbiosis with their host cells, known as endosymbionts. The \textit{Wolbachia} genus is a famous example of endosymbiotic bacteria infecting arthropod species. These bacteria are reproductive parasites which can be transmitted vertically through infection of the host female's eggs \cite{Knight2001}. Some \textit{Wolbachia} have altered the reproductive capabilities of their sexual host species to reproduce asexually by \Gls{parthenogenesis}. This effectively removes all males from the host population, benefitting the bacterium which can only be transmitted through females. In some species infection by \textit{Wolbachia} has even become necessary for reproduction. While the bacterium takes advantage of its host reproduction, it also provides numerous advantages such as resistance to viruses in flies and mosquitoes \citep{Hedges2008,Teixeira2008} and help with vitamin synthesis in bed bugs \cite{Nikoh2014}, illustrationg the blurry line between parasitism and mutualism.

In this work, we  focus on bacterial endosymbionts. Living directly inside of their host's cytoplasm, their genomic fate is most tightly linked to their host.

\section{The evolutionary context of intracellular parasitism}

The "arms race" is an analogy often used to describe the evolutionary dynamics of intracellular parasites with their hosts. Each organism evolves novel strategies to improve its own fitness at the expense of the other. This is the case for intracellular bacteria such as \textit{Legionella}, which can secrete a large arsenal of effector proteins into their host's cytoplasm. Many of these proteins are redundant in the sense that they interact with the same host proteins or pathways \cite{Ghosh2017}. Since perfectly redundant genes should be unstable due to genetic drift \cite{Bergthorsson2007}, their functions most likely have partial overlap, such as different affinity for certain substrates or ability to function in different pH or temperatures, and selective pressure is applied on these specificity. This is likely an important strategy for parasites with a broad host range or variable environments.

As obligate intracellular parasites become reliant on their host for most metabolic pathways, they undergo a process known as genome reduction: Pathways provided by the host need no longer be encoded by the parasite and are therefore lost \cite{McCutcheon2012}. This process eventually leads to the parasite becoming completely reliant on its host. The progressive accumulation of mutation (and loss) in their genome is known as \Gls{Muller}. The only way for intracellular parasites to escape this ineluctable degradation is to acquire genetic material, either from their host or from other microorganisms sharing the same host cytoplasm.

Such genetic transfers are known as horizontal gene transfer (HGT) and are a major contributor to bacterial genomes, with an estimated 80\% of genes being the product of HGT. More recently, HGT have also been detected in eukaryotes. Although they are much less frequent (about 1\% of genes), gene transfers from intracellular microorganisms to eukaryotic hosts are thought to have been involved in major shifts in environmental niche. Examples are the terrestrial colonization of plants and extremophile eukaryotes such as sea ice diatoms which acquired ice binding proteins from prokaryotes.

\section{Amoeba as a host model}

Free living amoeba are ubiquitous unicellular organisms found in bodies of water and soil. They feed on microorganisms by phagocytosis and are host to many parasites. They offer a great experimental model as many amoeba species are easy to grow in laboratory conditions and can be used for infection experiments.

Despite their extensive use as an infection model, only a few species have high quality genome assemblies available and their biology of free living amoeba is still largely unknown. For example, there is evidence for highly variable ploidy levels and horizontally acquired genes \cite{clarke2013}. Both of these features are likely important in their interactions with pathogens. It has been proposed that high ploidy levels are a mean for asexual amoeba to escape Muller's ratchet through homologous recombination between haplotypes \cite{Maciver2016}.

\section{\textit{Legionella pneumophila}}

\textit{L. pneumophila} infects a range of 15 species of amoebae and ciliated protozoa in the wild \cite{Rowbotham1980}, but it can also infect lung macrophages of humans and other mammalians.
There is no record of human to human \textit{L. pneumophila} transmission, making infection of macrophages an evolutionary dead-end for the bacterium. Its protozoan hosts act as reservoir and the \textit{L. pneumophila} is a major public health concern as it can contaminate water distribution and cause major outbreaks. When it is engulfed by a predatory cell, Legionella evades the lysosomal degradation route and survives in a "Legionella containing vacuole" (LCV). It does so using its type IV secretion system to secrete around 300 effector proteins into the host cytoplasm and rewire the host metabolic and signalling pathways. This stabilizes the pH, recruits nutrients and proteins towards the LCV and favors proliferation of \textit{L. pneumophila} in the host.

It was recently found that RomA, one of the effectors secreted by \textit{L. pneumophila} is a histone methyltransferase which can alter the histone methylation state throughout the host genome and affects the expression of a large number of genes \cite{Rolando2013}.

\section{\textit{Salmonella enterica}}

Unlike {L. pneumophila}, {S. enterica} only infects birds and mammals. It is a major human pathogen and the agent of typhoid fever.