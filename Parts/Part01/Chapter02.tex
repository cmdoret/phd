% Chapter X

\chapter{Infection through the lense of genomics} % Chapter title

\label{ch:01-02} % For referencing the chapter elsewhere, use \autoref{ch:name} 

%----------------------------------------------------------------------------------------

The toolset to detect and investigate bacterial infection traditionally included biochemical assays and microscopy. The recent technological advances in DNA sequencing have spurred a rapid extension of this toolset with NGS-derived methods. Here we introduce the different ways genomics can provide biological insights into the biology of bacterial pathogens.

\section{Pathogen characterization}

The most fundamental task related to infection in biomedical research is to detect the presence of infectious agents and characterize them. This allows to test patients presenting suspicious symptoms for the presence of known pathogens, or determine the pathogenicity of a particular strain. 

Genotyping was traditionally achieved using molecular biology techniques, such as \acrfull{RFLP} or \acrfull{PFGE} \cite{Ochoa-Díaz2018}. These techniques rely on the negative charge of the DNA molecules. When put in a gel submitted to an electromagnetic field, these molecules will migrate along the electrical current. The migration distance is proportional to the size of molecules. After migration is complete, the gel can be treated with chemical markers to highlight the location of DNA molecules. This will reveal discrete bands of similar-length DNA fragments. Together these bands form a bar-code which can be interpreted by the scientist to draw conclusion about the number and size of these fragments. In the case of \acrshort{RFLP}, the genome is prealably digested by \Gls{restriction enzyme}s. The digestion will result in a series of fragments whose lengths can be seen on the gel. Bacterial genotypes wil have different mutations which will affect the digestion pattern and resulting barcode on the gel.

While these methods work well to determine differences between alleles, they do not inform us on the actual DNA sequence involved. The advent of DNA sequencing made it possible to directly link phenotype with associated sequences of nucleotides. \acrfull{WGS} provides accurate information on an organism's genotype, down to down to the \acrfull{SNP}, allowing to define genotypes at a finer scale. The main shortcoming of \acrshort{WGS} is its higher cost than other genotyping techniques, but the recent plummeting of sequencing costs have made it releatively affordable. These advantages have made \acrshort{WGS} a popular approach in clinical settings.

\section{Genomics to probe homeostasis}

When host cells are exposed to or infected by a pathogen, their homeostatic state is disrupted. This disruption is a combination of alterations caused by the pathogen to colonize the host cell and host-triggered immune reactions to improve its survival. These two components can usually be unentangled in infection experiments by using a disabled pathogen. The pathogen will still harbour the antigens triggering host reactions, but will be unable to cause any harms. One can then deduce the pathogen-caused disruptions by comparing the infection results from real and disabled pathogen. 

Multiple levels of regulation are affected upon infection, from gene expression to epigenetic states, and over the years, a vast arsenal of NGS techniques have been developed to read these regulatory states. 

The most frequently used feature is gene expression. The transcribed RNAs present in the sample can be reverse-transcribed into cDNA and sequenced. The relative abundance of each gene's transcript allows to quantify the expression of the whole genome, known as the transcriptome. Transcriptomes can then be compared between different conditions to find out which genes undergo perturbations during infection. 

Epigenetic changes, in the form of chemical modification of histone proteins offer yet another way to regulate gene expression in eukaryotes. The amount of these epigenetic marks can be measured along the genome using another NGS-derived technique known as \acrfull{ChIPseq}. In \acrshort{ChIPseq}, the chromatin sample is crosslinked with formaldehyde to generate covalent bonds between proteins and DNA. The sample is then sonicated to break the DNA into smaller fragments. Beads coated with specific antibodies against a protein of interest (e.g. an epigenetic mark) are then added and the beads are then precipitated to retrieve them. The crosslinked is then reversed and the DNA fragments purified. This allows to retrieve all genomic regions that were bound to the protein of interest. 


\section{Capturing chromosome conformation}

Although DNA is a linear (or circular) molecule, it can fold back on itself and form three-dimensional structures which have several benefits compared to a linear structure. These benefit include compactness: For example, the human chromosome 1 would be 85mm long if straightened, but the whole genome fits in a nucleus where the diameter is around 10$\mu m$. Another key feature of genome folding is that it can be used as a higher order way to regulate gene expression. Compacting large regions of the genome by spreading of heterochromatin allows to downregulate their activity. Smaller scale structures allow to fine tune gene regulation more locally. For example, chromatin loops can bring enhancer and promoters in close spatial proximity even if they would otherwise be far apart on the linear sequence. Compact chromatin domain also form local neighbourhoods where different loci are in close spatial proximity, while loci in distinct domains are isolated from each other. All these levels of spatial regulation are important to understand the coordination of the gene expression programme with other celluar processes.

The use of genomics to investigate the three-dimensional organisation of the genome started with the invention of \acrfull{3C} \cite{Dekker2002}. This technique allowed to measure the frequency of physical interactions a pair of loci. This is done by crosslinking the genome with formaldehyde, which forms stable bonds between DNA and proteins, and subsequently digesting the genome with a restriction enzyme. The digested genome is then relilgated, and the religation will happen with different neighbouring fragment. Loci which are closer in space will be religated more often with each other in the population of cells. The crosslink is then reverted and qPCR is used to measure the quantity of religated products containing the two loci of interest. 

Since then, many derivative of the \acrshort{3C} technique have been developed. The most significant improvement was brought by Hi-C. This method works similarly to 3C except that next generation sequencing is used instead of qPCR. This allows to quantify the interaction frequency of all versus all loci in the genome instead of using specific primers for a pair of locus. Hi-C also has an additional step where biotinylated bases are added during religation. This allows to pull-down religation products using streptavidin so that only products that underwent the digestion-religation process are sequenced.

Hi-C allows to generate interaction frequency maps of the whole genome wich reflect its 3D structure.

\section{Combining layers of biological informations}

When analysing the different layers of genome regulations (i.e. gene expression, epigenetics, spatial interactions), one of the main challenges is to find a efficient way to combine these informations. 

More often than not, they are analysed separately to find regions of deregulation common to the different layers. But there have already been attempts at fully integrating these levels of informations %MOFA, DL from Anshul Kundaje


