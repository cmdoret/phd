% Chapter X

\chapter{Thesis objectives} % Chapter title

\label{ch:01-04} % For referencing the chapter elsewhere, use \autoref{ch:name} 

Throughout this first part, we have laid out the scope of host-pathogen interactions and summarized the current state of genomics in relation to regulation and 3D genomes. Genomics is a fast changing field and there is a need for computational tools to extract meaningful biological information from the wealth of data.

Throughout the next part, we will introduce our contributions to the field and main results. In the first chapter, we explain our methodological developments related to chromosome conformation capture technologies. In the second chapter, we will present our chromosome scale genome assembly of \textit{A. castellanii}. We then use this resource for our main findings on the genomic changes happening during infection by \textit{L. pneumophila}. In the last chapter we will focus on murine bone macrophages infection by \textit{S. enterica} and the genomic alterations it entails. We will end with part 3 where we discuss various aspects of genomics in infection biology, including prospects and limitations.

% Finally, in chapter 4, we will discuss additional results related to the implications of viral integrations linked to hepatocellular carcinoma in the human genome. 

In this work, we develop accessible and performant methods to extract information from 3C technologies and use them to identify changes happening during infections in various organisms. We then use external data such as gene expression to assess the genes involved in those alterations and discuss how they could be associated with the infection process.
%----------------------------------------------------------------------------------------

