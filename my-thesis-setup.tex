%%%%%%%%%%%%%%%%%%%%%%%%%%%%%%%%%%%%%%%%%
% Thesis Configuration File
%
% Important note:
% The main lines to change in this file are in the DOCUMENT VARIABLES
% section, the rest of the file is for advanced configuration.
%
%%%%%%%%%%%%%%%%%%%%%%%%%%%%%%%%%%%%%%%%%
% !TEX root = my-thesis.tex


% **************************************************
% Files' Character Encoding
% **************************************************
\PassOptionsToPackage{utf8}{inputenc}
\usepackage{inputenc}

% **************************************************
% Silence annoying warnings
% **************************************************
\RequirePackage{silence}
\WarningFilter{titlesec}{Non standard sectioning command}
\WarningFilter{scrreprt}{Usage of package}
\WarningFilter{scrreprt}{Activating an ugly workaround}
\WarningFilter{titlesec}{Non standard sectioning command detected}

% **************************************************
% Information and Commands for Reuse
% **************************************************
\newcommand{\thesisTitle}{Exploring the genomic complexity of bacterial infection in 3D}
\newcommand{\thesisName}{Cyril Matthey-Doret}
\newcommand{\thesisSubject}{Thèse de doctorat en Bioinformatique}
\newcommand{\thesisDate}{1er Octobre 2021}
\newcommand{\thesisVersion}{0.1}

\newcommand{\thesisPresident}{Pr. X}
\newcommand{\thesisPresidentUniversity}{\protect{Université X}}
\newcommand{\thesisPresidentDepartment}{Departement X}

\newcommand{\thesisFirstReviewer}{Dr. A}
\newcommand{\thesisFirstReviewerUniversity}{\protect{Université A}}
\newcommand{\thesisFirstReviewerDepartment}{Département A}

\newcommand{\thesisSecondReviewer}{Dr. B}
\newcommand{\thesisSecondReviewerUniversity}{\protect{Université B}}
\newcommand{\thesisSecondReviewerDepartment}{Département B}

\newcommand{\thesisFirstExaminer}{Dr. C}
\newcommand{\thesisFirstExaminerUniversity}{\protect{Université C}}
\newcommand{\thesisFirstExaminerDepartment}{Département C}

%\newcommand{\thesisSecondExaminer}{Dr. John Doe}
%\newcommand{\thesisSecondExaminerUniversity}{\protect{Clean Thesis Style University}}
%\newcommand{\thesisSecondExaminerDepartment}{Department of Clean Thesis Style}

\newcommand{\thesisFirstSupervisor}{Pr. Romain Koszul}
\newcommand{\thesisSecondSupervisor}{Dr. D}

\newcommand{\thesisUniversity}{\protect{Sorbonne Université}}
\newcommand{\thesisUniversityDepartment}{École doctorale Complexité du Vivant}
\newcommand{\thesisUniversityInstitute}{Institut Pasteur}
\newcommand{\thesisUniversityGroup}{Unité de Régulation Spatiale des Génomes}
\newcommand{\thesisUniversityCity}{Paris}
\newcommand{\thesisUniversityCityCedex}{Cedex 15}
\newcommand{\thesisUniversityStreetAddress}{25-28 Rue du Docteur Roux}
\newcommand{\thesisUniversityPostalCode}{75724}


% **************************************************
% Debug LaTeX Information
% **************************************************
%\listfiles


% **************************************************
% Packages options
% **************************************************

\PassOptionsToPackage{% setup clean thesis style
    figuresep=colon,%
    sansserif=false,%
    hangfigurecaption=false,%
    hangsection=true,%
    hangsubsection=true,%
    colorize=full,%
    colortheme=blueroyalred,%
    bibsys=bibtex,%
%     bibfile=Mendeley,%
    bibfile=library,%
    bibstyle=numeric,%
    wrapfooter=true,%
}{cleanthesis}

\PassOptionsToPackage{
	acronyms,
    translate=babel,
    xindy,
    toc,
    nopostdot,
    entrycounter,
    nohypertypes={acronym,notation},
    indexonlyfirst
}{glossaries}

\PassOptionsToPackage{
	english,
	french,
    main=english
}{babel}

\PassOptionsToPackage{
	user,
    counter,
    hyperref
}{zref}

\PassOptionsToPackage{
	nameinlink
}{cleveref}

\PassOptionsToPackage{
	para,
    online,
    flushleft
}{threeparttable}

% Remove several fields in bibliography



% **************************************************
% Load and Configure Packages
% **************************************************


\usepackage{algorithm}
\usepackage{algorithmic}
\usepackage{babel} 						% babel system, adjust the language of the content
\usepackage{cleanthesis}
\usepackage{glossaries}						% Allow glossary and acronyms
\usepackage[amssymb]{SIunits}				% symboles unites SI
\usepackage{chngcntr}						% Reset Chapter counter for each part
\usepackage{amsmath}
\usepackage[euler]{textgreek}				% For better greek letters in text mode
\usepackage{subcaption}						% Allow subfigures
% \usepackage[lofdepth,lotdepth]{subfig}
% \usepackage{etoolbox}
% \usepackage{zref}							% Better label tags
% \usepackage{cleveref}						% Better autorefs
% \usepackage{xparse}
\usepackage{pdfpages}
\usepackage{xassoccnt}
\usepackage[user,counter]{zref}
\usepackage{xparse}
\usepackage{threeparttable}
\usepackage{booktabs}
\usepackage{tabularx}
\usepackage{colortbl}
\usepackage{tabularx}
\usepackage{multirow}
\usepackage{listings}
\usepackage{wrapfig}
\usepackage{array}
% \usepackage{nameref}
\usepackage{marginnote}
%\usepackage{minted}
\usepackage{tikz}
\usetikzlibrary{shapes,positioning,arrows}

% **************************************************
% Setup
% **************************************************
% ..................................................
% Commands
% ..................................................
\newsavebox{\largestimage}

\DeclareCiteCommand{\citeauthorfirstlast}
  {\boolfalse{citetracker}%
   \boolfalse{pagetracker}%
   \DeclareNameAlias{labelname}{first-last}%
   \usebibmacro{prenote}}
  {\ifciteindex
     {\indexnames{labelname}}
     {}%
   \printnames{labelname}}
  {\multicitedelim}
  {\usebibmacro{postnote}}
% ref commands, e.g. for images, tables and text labels
% --------------------------------------------------
% RESULT = (siehe Tab. 12.4)
\newcommand{\tabref}[1]{(see Tab.~\ref{#1})}
%
% RESULT = (siehe Tab. 12.4)
\newcommand{\tableref}[1]{(see Tab.~\ref{#1} Page~\pageref{#1})}
%
% --------------------------------------------------
% RESULT = (siehe 3.4)
\newcommand{\tref}[1]{(see \ref{#1})}
%
% RESULT = Abschnitt 3.4
\newcommand{\treft}[1]{Section~\ref{#1}}
%
% RESULT = (siehe 3.4, Seite 12)
\newcommand{\textref}[1]{(see \ref{#1}, Page~\pageref{#1})}
%
% RESULT = Abschnitt 3.4 (siehe Seite 12)
\newcommand{\textreft}[1]{Section~\ref{#1} (see Page~\pageref{#1})}
%
% --------------------------------------------------
% RESULT = (siehe Abb. 10.4)
\newcommand{\fref}[1]{(see Fig.~\ref{#1})}
%
% RESULT = (siehe Abb. 10.4 b)
\newcommand{\frefadd}[2]{(see Fig.~\ref{#1}~#2)}
%
% RESULT = (siehe Abb. 10.4, Seite 12)
\newcommand{\figref}[1]{(see Fig.~\ref{#1}, Page~\pageref{#1})}
%
% RESULT = (siehe Abb. 10.4 b, Seite 12)
\newcommand{\figrefadd}[2]{(see Fig.~\ref{#1}~#2, Page~\pageref{#1})}
%
% RESULT = Abbildung 10.4
\newcommand{\figreft}[1]{Figure~\ref{#1}}
%
% RESULT = Abbildung 10.4 b
\newcommand{\figrefaddt}[2]{Figure~\ref{#1}~#2}
%
% --------------------------------------------------
% RESULT = (siehe Seite 12)
\newcommand{\seepage}[1]{(see Page~\pageref{#1})}
% ..................................................
% Hyperref
% ..................................................
\hypersetup{% setup the hyperref-package options
    pdftitle={\thesisTitle},    %   - title (PDF meta)
    pdfsubject={\thesisSubject},%   - subject (PDF meta)
    pdfauthor={\thesisName},    %   - author (PDF meta)
    plainpages=false,           %   -
    colorlinks=true,            %   - colorize links?
    linktocpage=true,
    pdfborder={0 0 0},          %   -
    breaklinks=true,            %   - allow line break inside links
    bookmarksnumbered=true,     %
    bookmarksopen=true,         %
    linkcolor=ctcolorblue,
    urlcolor=ctcolorblue,
    citecolor=black,
}


% ..................................................
% Counter labels
% ..................................................
\counterwithin*{chapter}{part}	% Reset chapter counter each time part counter is called
\counterwithin{table}{part}
\counterwithout{table}{chapter}
\counterwithin{figure}{part}
\counterwithout{figure}{chapter}
\renewcommand{\thetable}{\thepart.\Alph{table}}
\renewcommand{\thefigure}{\thepart.\Alph{figure}}
\renewcommand*{\figureformat}{\figurename~\thefigure}
\renewcommand*{\tableformat}{\tablename~\thetable}
% ..................................................
% Lists
% ..................................................
%\frenchbsetup{StandardLists=true}
\setlist[itemize]{label={\LARGE\raisebox{-0.3ex}{\textbullet}}, font=\color{ctcoloroyalblue}}
\newlist{todolist}{itemize}{3}
\setlist[todolist]{label={\LARGE\raisebox{-0.3ex}{\textbullet}},font=\color{color_todo}\bfseries,itemsep=0pt,parsep=0pt,before=\color{color_todo}}
% ..................................................
% Glossaries  
% ..................................................
\DeclareDocumentCommand{\newdualentry}{ O{} O{} m m m m } {
  \newglossaryentry{gls-#3}{name={#5},text={#5\glsadd{#3}},
    description={#6},#1
  }
  \makeglossaries
  \newacronym[see={[Glossary:]{gls-#3}},#2]{#3}{#4}{#5\glsadd{gls-#3}}
}
\defglsentryfmt{%
  \ifglsused{\glslabel}{%
    \glsgenentryfmt\ignorespaces%
  }{%
    % Typeset first use
    \textit{\glsgenentryfmt}\ignorespaces%
  }%
}
\renewcommand*{\glsentrycounterlabel}{}%
% \renewcommand*{\glspostlinkhook}{\textsuperscript{\ref{glsentry-\glslabel}}}
% \renewcommand*{\glspostlinkhook}{\textsuperscript{\glsrefentry{\glslabel}}}
% \renewcommand*{\glstextformat}[1]{\textcolor{black}{#1}}		% Hack to remove color links in glossaries
\renewcommand{\glossarypreamble}{\glsresetentrycounter} 
\newglossary[nlg]{notation}{not}{ntn}{Notation}
% \defglsdisplayfirst[\glsdefaulttype]{\textit{#1}\ignorespaces}

% ..................................................
% Bibliography
% ..................................................
\defbibheading{bibliography}[\bibname]{\chapter*{#1}\markboth{#1}{#1}}	% Redefine heading/footer for bibliography

% ..................................................
% Subcaption
% ..................................................
\renewcommand{\thesubfigure}{\alph{subfigure}}
\captionsetup[subfigure]{labelformat=simple, labelsep=none, justification=centering}
% \DeclareCaptionLabelFormat{graysf}{\textcolor{gray}{\sffamily (#2)}}
% \captionsetup[subfigure]{labelsep=space,labelformat=graysf}
% ..................................................
% Caption names
% ..................................................
\renewcaptionname{english}{\figurename}{Fig.}
\renewcaptionname{english}{\tablename}{Tab.}
%\renewcaptionname{french}{\partname}{Partie}
%\renewcaptionname{french}{\tablename}{Tableau}
% ..................................................
% Zref 
% ..................................................
% TODO: Sounds good, doesn't work !!
\makeatletter

% Define new properties
\zref@newprop{childcountervalue}{\arabic{\LastRefSteppedCounter}}% This is the naked value
\zref@newprop{parentcountervalue}{\csname the\GetParentCounter{\LastRefSteppedCounter}\endcsname}
\newenvironment{conditions}
{\par\vspace{\abovedisplayskip}\noindent\begin{tabular}{>{$}l<{$} @{${}={}$} l}}
	{\end{tabular}\par\vspace{\belowdisplayskip}}
% Add the new properties to the main property list stored with \zlabel
\zref@addprops{main}{childcountervalue,parentcountervalue}


\NewDocumentCommand{\parentref}{m}{%
  \zref@ifrefundefined{#1}{%
    -100%
  }{%
    \zref@extract{#1}{parentcountervalue}%
  }%
}


\makeatother

% \GetAllResetLists% Important
% \RegisterPostLabelHook{\zlabel}% Important

% ..................................................
% Color tables 
% ..................................................
% Wener post (https://tex.stackexchange.com/a/33761)
% David Carlisle post at tex.exchange (https://tex.stackexchange.com/a/91498)
\colorlet{blcolor}{gray!80}
\colorlet{tableheadcolor}{gray!25} % Table header colour = 25% gray
\colorlet{tablerowcolor}{gray!10} % Table row separator colour = 10% gray
\newcommand{\headcol}{\rowcolor{tableheadcolor}} %
\newcommand{\rowcol}{\rowcolor{tablerowcolor}} %
% Command \topline consists of a (slightly modified) \toprule followed by a \heavyrule rule of colour tableheadcolor (hence, 2 separate rules)
\newcommand{\topline}{\arrayrulecolor{black}\specialrule{0.1em}{\abovetopsep}{0pt}%
            \arrayrulecolor{tableheadcolor}\specialrule{\belowrulesep}{0pt}{0pt}%
            \arrayrulecolor{black}}
% Command \midline consists of 3 rules (top colour tableheadcolor, middle colour black, bottom colour white)
\newcommand{\midline}{\arrayrulecolor{tableheadcolor}\specialrule{\aboverulesep}{0pt}{0pt}%
            \arrayrulecolor{black}\specialrule{\lightrulewidth}{0pt}{0pt}%
            \arrayrulecolor{white}\specialrule{\belowrulesep}{0pt}{0pt}%
            \arrayrulecolor{black}}
% Command \rowmidlinecw consists of 3 rules (top colour tablerowcolor, middle colour black, bottom colour white)
\newcommand{\rowmidlinecw}{\arrayrulecolor{tablerowcolor}\specialrule{\aboverulesep}{0pt}{0pt}%
            \arrayrulecolor{black}\specialrule{\lightrulewidth}{0pt}{0pt}%
            \arrayrulecolor{white}\specialrule{\belowrulesep}{0pt}{0pt}%
            \arrayrulecolor{black}}
% Command \rowmidlinewc consists of 3 rules (top colour white, middle colour black, bottom colour tablerowcolor)
\newcommand{\rowmidlinewc}{\arrayrulecolor{white}\specialrule{\aboverulesep}{0pt}{0pt}%
            \arrayrulecolor{black}\specialrule{\lightrulewidth}{0pt}{0pt}%
            \arrayrulecolor{tablerowcolor}\specialrule{\belowrulesep}{0pt}{0pt}%
            \arrayrulecolor{black}}
% Command \rowmidlinew consists of 1 white rule
\newcommand{\rowmidlinew}{\arrayrulecolor{white}\specialrule{\aboverulesep}{0pt}{0pt}%
            \arrayrulecolor{black}}
% Command \rowmidlinec consists of 1 tablerowcolor rule
\newcommand{\rowmidlinec}{\arrayrulecolor{tablerowcolor}\specialrule{\aboverulesep}{0pt}{0pt}%
            \arrayrulecolor{black}}
% Command \bottomline consists of 2 rules (top colour
\newcommand{\bottomline}{\arrayrulecolor{white}\specialrule{\aboverulesep}{0pt}{0pt}%
            \arrayrulecolor{black}\specialrule{\heavyrulewidth}{0pt}{\belowbottomsep}}%
\newcommand{\bottomlinec}{\arrayrulecolor{tablerowcolor}\specialrule{\aboverulesep}{0pt}{0pt}%
            \arrayrulecolor{black}\specialrule{\heavyrulewidth}{0pt}{\belowbottomsep}}%
% Middle line connecting heading row and the second row
\newcommand{\rowmidlineHR}{\arrayrulecolor{tableheadcolor}
  \specialrule{\aboverulesep}{0pt}{0pt}%
  \arrayrulecolor{black}\specialrule{\lightrulewidth}{0pt}{0pt}%
  \arrayrulecolor{tablerowcolor}\specialrule{\belowrulesep}{0pt}{0pt}%
  \arrayrulecolor{black}}
  % Command \rowmidlinewc consists of 3 rules
  % (top colour tableheadcolor, middle colour black, bottom colour tablerowcolor)
% Secondary gray middle line 
\newcommand{\rowmidlineG}{\arrayrulecolor{tablerowcolor}%
  \specialrule{\aboverulesep}{0pt}{0pt}%
  \arrayrulecolor{blcolor}\specialrule{\lightrulewidth}{0pt}{0pt}%
  \arrayrulecolor{tablerowcolor}\specialrule{\belowrulesep}{0pt}{0pt}%
  \arrayrulecolor{black}}
\renewcommand{\arraystretch}{1.3}
\newcolumntype{Y}{>{\centering\arraybackslash}X}
\newcolumntype{C}{>{\centering\arraybackslash}c}
\newcolumntype{M}{>{\centering\arraybackslash}m}
% ..................................................
% Listings
% ..................................................
\lstdefinelanguage{Ini}
{
	basicstyle=\ttfamily\small,
	columns=fullflexible,
	morecomment=[s][\color{Orchid}\bfseries]{[}{]},
	morecomment=[l]{\#},
	morecomment=[l]{;},
	commentstyle=\color{gray}\ttfamily,
	morekeywords={},
	otherkeywords={=,:},
	keywordstyle={\color{green}\bfseries}
}


%\newenvironment{conditions}
%{\par\vspace{\abovedisplayskip}\noindent\begin{tabular}{>{$}l<{$} @{${}={}$} l}}
%	{\end{tabular}\par\vspace{\belowdisplayskip}}
