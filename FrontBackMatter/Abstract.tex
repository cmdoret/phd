% !TEX root = ../thesis-example.tex
%
\pdfbookmark[0]{\abstractname}{abstract}
\chapter*{\abstractname}
\label{sec:abstract}
\vspace*{-10mm}

Numerous bacteria and viruses use cells from another species to ensure their proliferation. This mode of reproduction implies the pathogen must escape the host immune system and reprogram its metabolism to sustain its own needs. These changes are often detrimental to the host cell and cause pathologies or death. The intracellular bacteria which use this mode of operation have been the focus of many studies aiming to understand their "hijacking" mechanisms. Recent advances in genomics have largely stimulated research in this field by offering the possibility to decipher the sequence of genes expressed during infection. Several intracellular bacteria secrete "effector" proteins into the host cytoplasm which interact with its proteins and affect its genetic expression program. Recently, studies in \textit{Legionella pneumophila}, an experimental model for intracellular bacteria, have shown it was able to alter the epigenetic state of its host. Such modifications allow rapid physiological changes and are intimately linked to the spatial organisation of the genome. 3D genome organisation plays an important part in many biological processes, for example by modulating gene expression through long range interactions in the sequence. Throughout this work, we develop computational tools to explore and measure spatial changes occuring in the genome, and exploit them to investigate the changes taking place during infection by intracellular bacteria. We use the model species \textit{Legionella pneumophila} and \textit{Salmonella enterica} to explore structural changes taking place in the host chromosomes and their link with genetic expression.

\vspace*{20mm}

\chapter*{Résumé}
\label{sec:abstract-diff}

De nombreuses bactéries et virus utilisent les cellules d'autres espèces pour assurer leur prolifération. Ce mode de reproduction implique que le pathogène doit échapper au système immunitaire de son hôte et reprogrammer son métabolisme pour subvenir à ses propres besoins. Ces changements s'opèrent souvent au détriment de la cellule hôte et causent des pathologies ou la mort. Les bactéries intracellulaires utilisant ce mode de fonctionnement et font l'objet de nombreuses études qui visent à comprendre leurs mécanismes de "piratages". Les récents progrès en génomique ont largement stimulé la recherche dans ce domaine en offrant la possibilité de déchiffrer la séquence des gènes exprimés durant l'infection. De nombreuses bactéries intracellulaires sécrètent des "effecteurs" dans le cytoplasme de leur hôte qui vont interagir avec ses protéines et affecter son programme d'expression génétique. Récemment des études dans la légionelle (\textit{Legionella pneumophila}), un modèle expérimental pour les bactéries intracellulaires, ont démontré qu'elle était capable d'altérer la régulation épigénétique de son hôte. Ce genre de modifications permet des changements physiologiques rapides et est intimement lié à l'organisation spatiale du génome. L'organisation 3D du génome joue un rôle important dans de nombreux processus biologiques, par exemple en modulant l'expression génétique par la formation d'interaction entre des éléments éloignés dans la séquence d'ADN. A travers ce travail, nous développons des outils computationels pour explorer et mesurer les changements spatiaux du génome, et nous les exploitons pour étudier les changements qui ont lieu pendant l'infection par des bactéries intracellulaires. Nous utilisons en particulier les modèles \textit{Salmonella enterica } et \textit{Legionella pneumophila} pour explorer les changements de structure qui surviennent dans les chromosomes de leurs hôtes et leurs liens avec l'expression génétique.
